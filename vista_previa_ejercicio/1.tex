\documentclass{article}%
\usepackage[T1]{fontenc}%
\usepackage[utf8]{inputenc}%
\usepackage{lmodern}%
\usepackage{textcomp}%
\usepackage{lastpage}%
\usepackage{enumitem}%
%
%
%
\begin{document}%
\normalsize%
\section*{Vista Previa Ejercicio \#1}%
\subsection*{Ejercicio 1.}%
La función $$F(x)= \int_{-1}^{x}{(|1-x|-x)dx}$$ admite para $x \leq 1.1$ una expresión $F(x)=ax^2+bx + c$, en forma de un polinomio de grado menor o igual a 2. \linebreak 
\linebreak 
Determinar la suma a + b + c de los coeficientes de ese polinomio.%
\begin{enumerate}[label=\alph*)]%
\item%
Ningúna de las demás es correcta%
\item%
-34.98%
\item%
-3.28%
\item%
-33.88%
\end{enumerate}

%
\end{document}